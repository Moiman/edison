\documentclass{article}

\usepackage{hcar}

\begin{document}

\begin{hcarentry}{Edison}
\report{Rob Dockins}
\status{stable, maintained}
\makeheader

Edison is a library of purely function data structures for Haskell
originally written by Chris Okasaki.  Conceptually, it consists of two
things:
\begin{enumerate}
\item A set of type classes defining data the following data structure
      abstractions: ``sequences'', ``collections'' and ``associative collections''
\item Multiple concrete implementations of each of the abstractions.
\end{enumerate}


In theory, either component may be used independently of the other.

I took over maintenance of Edison about 18 months ago
in order to update Edison to use the most current Haskell tools.
The following major changes have been made since version 1.1, which
was released in 1999.

\begin{itemize}
\item Typeclasses updated to use fundeps (by Andrew Bromage)
\item Implementation of ternary search tries (by Andrew Bromage)
\item Modules renamed to use the hierarchical module extension
\item Documentation haddockized
\item Source moved to a darcs repository
\item Build system cabalized
\item Unit tests integrated into a single driver program which exercises
      all the concrete implementations shipped with Edison
\item Multiple additions to the APIs (mostly the associated collection API)
\end{itemize}


Edison is currently in maintain-only mode. I don't have the time required
to enhance Edison in the ways I would like.  If you are interested in
working on Edison, don't hesitate to contact me.

The biggest thing that Edison needs is a benchmarking suite.  Although
Edison currently has an extensive unit test suite for testing correctness,
and many of the data structures have proven time bounds,
I have no way to evaluate or compare the quantitative
performance of particular data structure implementations in a
principled way.  Unfortunately, benchmarking data structures
(especially in a non-strict language) is difficult to do well.
If you have an interest or experience in this area, your help
would be very much appreciated.

\FurtherReading
\begin{itemize}
\item \url{http://http://www.cs.princeton.edu/~rdockins/edison/home/}
\end{itemize}
\end{hcarentry}

\end{document}
